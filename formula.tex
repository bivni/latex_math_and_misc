
% !TeX root = main_2020_10_31.tex

	\section{Формулы}
	
%  \DeclareMathOperator{\covarince}{cov} -- в преамбуле
	
	Зарплаты смотрите на странице \pageref{tab:salary} в  таблице \ref{tab:salary}
	
	
	На странице \pageref{eq:sum1} смотрите формулу \ref{eq:sum1}  для   суммы.
	
	Строчная формула (inline equation) $ \rho=\frac{m}{V}$
	
	
	Выключная формула (display equation) в стиле \TeX 
	
	 $$ \rho=\frac{m}{V}$$
	 
	 
		Выключная формула (display equation) в стиле \LaTeX 
	
	\[ S^2 = a_1^2+ a_2^2+ a_3^2+\dots+a_n^2\] 
	
	\begin{equation}
		S^2=\sum_{i=1}^{n} a_i^2  \label{eq:sum1}
	\end{equation}
	
	
	\begin{equation*}
	S^2=\sum_{i=1}^{n} a_i^2
	\end{equation*}
	
	
    \begin{equation*}
    S^4=\left(\sum_{i=1}^{n} a_i^2\right)^2
    \end{equation*}
    	
    	
    	\begin{equation}
    		r=\frac{\covarince x y}{\sigma_x\cdot \sigma_y}
    	\end{equation}
    	
  
  \begin{equation}
  r^2=\left(\frac{\covarince x y}{\sigma_x\cdot \sigma_y}\right)^2
  \end{equation}
  
  
  \begin{equation}
     \int_{a}^b f(x) dx  = \left.F(x)\right|_a^b
  \end{equation}
    	
    	
  \begin{eqnarray}
  	\sin x &=& \cos y \\
  	\sin x + \cos 2y& =&\sin 2x +\cos y	
  \end{eqnarray}  	
 
 
\begin{equation}
 \left\{\begin{array}{@{}ccc}
\sin x &=& \cos y \\
\sin x + \cos 2y& =&\sin 2x +\cos y	
\end{array} \right. 	
\end{equation} 
 